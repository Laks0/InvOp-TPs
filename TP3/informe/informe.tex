\documentclass[10pt]{article}
\usepackage[a4paper, margin=1in]{geometry}
\usepackage[utf8]{inputenc}
\usepackage[spanish]{babel}
\usepackage{caratula}
\usepackage{amsmath}
\usepackage{amssymb}
\usepackage{hyperref}
\usepackage{enumitem}
\usepackage{graphicx}
\usepackage{subcaption} % Para subfiguras
\usepackage{xcolor}

\hypersetup{
    colorlinks=true,
    linkcolor=blue,
    filecolor=magenta,      
    urlcolor=cyan,
    pdftitle={Overleaf Example},
    pdfpagemode=FullScreen,
	}
	
\setlength{\parskip}{1em}   % Espacio vertical entre párrafos

\begin{document}

	\titulo{TP3}

	\fecha{\today}

	\materia{Introducción a la Investigación Operativa y Optimización}

	\integrante{Laks, Joaquín}{425/22}{laksjoaquin@gmail.com}
	\integrante{Szabo, Jorge}{1683/21}{jorgecszabo@gmail.com}
	\integrante{Wilders Azara, Santiago}{350/19}{santiago199913@gmail.com}

	\maketitle

\section{Introducción}

En este trabajo práctico se implementaron distintos algoritmos para encontrar la ubicación óptima de un centro de servicio médico que responda a zonas afectadas con distintos niveles de atención necesarios. Mas formalmente, se busca encontrar un punto que minimice la suma de las distancias euclidianas ponderadas respecto a un conjunto discreto de puntos, es decir, encontrar la mediana geométrica de dicho conjunto.

\section{Algoritmos}


\subsection{Weiszfeld}


\subsection{Método de Hooke y Jeeve}
	
	
\subsection{Descenso de gradiente}
	

\section{Comparación de tiempos entre algoritmos}


\subsection{Conclusión}

\end{document}
