\documentclass[10pt,a4paper]{article}
\usepackage[utf8]{inputenc}
\usepackage[spanish]{babel}
\usepackage{a4wide} % márgenes un poco más anchos que lo usual
\usepackage{caratula}
\usepackage{amsmath}

\begin{document}

\titulo{TP1}

\fecha{\today}

\materia{Introducción a la Investigación Operativa y Optimización}

\integrante{Laks, Joaquín}{425/22}{laksjoaquin@gmail.com}
\integrante{Szabo, Jorge}{1683/21}{jorgecszabo@gmail.com}
\integrante{Wilders Azara, Santiago}{350/19}{santiago199913@gmail.com}

\maketitle

\section*{Datos}

\begin{center}
\textbf{Horas por 1000 litros de combustible}

\begin{tabular}{| c || c | c | c |}
	\hline
						& Refinado & Fraccionado & Embalaje \\
	\hline
	Aviones   & 10       & 20          & 4 \\
	\hline
	Vehículos & 5        & 10          & 2 \\
	\hline
	Keronsene & 3        & 6           & 1 \\
	\hline
\end{tabular}

\vspace{5mm}
\textbf{Tiempos y gastos fijos}

\begin{tabular}{| c || c | c |}
	\hline
										& Capacidad Mensual & Gasto Fijo \\
	\hline
	Refinado           & 38.000 horas      & \$5.000.000 \\
	\hline
	Fraccinoado        & 80.000 horas      & \$5.000.000 \\
	\hline
	Embalaje Aviones   & 4.000 horas       & \$2.000.000 \\
	\hline
	Embalaje Vehículos & 6.000 horas       & \$1.000.000 \\
	\hline
	Embalaje Keronsene & 7.000 horas       & \$500.000 \\
	\hline
\end{tabular}

\vspace{5mm}

	\textbf{Costos y ganancias variables por 1000 litros}

	\begin{tabular}{| c || c | c | c | c | c | c|}
		\hline
							& Precio de venta & Materia prima & Refinado & Fraccionado & Embalaje &
							Ganancia\\
		\hline
		Aviones   & \$16.000        & \$4.000       & \$4.100  & \$1.000   & \$1.000  & \$5.900 \\
		\hline
		Vehículos & \$8.000         & \$1.000       & \$3.000  & \$600     & \$500 & \$2.900 \\
		\hline
		Keronsene & \$4.000         & \$500         & \$1.500  & \$400     & \$400  & \$1.200 \\
		\hline
	\end{tabular}
\end{center}

\section{}
\textbf{Calcular la ganancia o pérdida (prorrateando los gastos fijos) de cada producto que se obtuvo en el mes
anterior (cuando se produjeron 500.000 litros de combustible para aviones, 3.000.000 de combustible para
vehículos y 6.000.000 litros de kerosene) y la ganancia (o pérdida) total de la compañía.}

\vspace{5mm}

Los gastos de refinado y fraccionado van a ser prorrateados entre los tres combustibles producidos. Para un tipo de combustible, se modela la producción con variables $X_i$ para la producción de mil litros de combustible con $i \in  \{a,v,k\}$ para aviones, vehículos y kerosene respectivamente.

Teniendo en cuenta los costos variables y los precios de venta por cada 1000 litros de combustible, definimos $G_i$ la ganancia variable de dicha cantidad.

Luego los costos fijos $F_j$ mensuales de refinado y fraccionado se comparten para los tres tipos de combustible. Siendo ademas $H_{ij}$ el tiempo de producción de 1000 litros de combustible $i$ en el proceso $j$, con $j \in \{r, f\}$. El costo para producir $X_i$ miles de litros de combustible $i$ en un proceso $j$ se calcula como:

$$
\frac{F_j H_{ij} X_i}{H_{aj}X_a + H_{vj}X_v + H_{kj}X_k}
$$

Las etapa de embalaje es independiente para cada tipo de combustible. Notamos $E_i$ con $i \in \{a,v,k\}$ el costo fijo de operar el sector de embalaje para el combustible $i$.

Finalmente, el balance final se obtiene de restar a las ganancias variables, los costos prorroteados y costos de embalaje para cada tipo de combustible $i$, es decir:
$$
G_i X_i - \sum_{j\in\{r,f\}}{\frac{F_j H_{ij} X_i}{H_{aj}X_a + H_{vj}X_v + H_{kj}X_k}} \quad  - E_i
$$

\clearpage

\begin{center}
	\textbf{Balances finales}
	\vspace{3mm}

	\begin{tabular}{| c | c |}
		\hline
		&           Balance    \\
		\hline
		Aviones   & -\$365.790  \\
		\hline
		Vehículos & \$3.752.632 \\
		\hline
		Kerosene  & \$1.963.158 \\
		\hline
		Total     & \$5.350.000\\
		\hline
	\end{tabular}
\end{center}

Vemos que la empresa en total da ganancia, pero el combustible para aviones pérdida.
\section{}
\textbf{Si la empresa no hubiese producido combustible para aviones manteniendo en los mismos valores los otros productos, ¿la ganancia de la compañía habría sido mejor? Suponer que se cierra el sector de embalaje de combustibles para aviones.}

Haciendo la cuenta, nos quedaría la siguiente tabla:

\begin{center}
	\textbf{Ganancias totales}
	\vspace{3mm}
	
	\begin{tabular}{| c | c |}
		\hline
		&           Ganancia    \\
		\hline
		Aviones   & \$0  \\
		\hline
		Vehículos & \$3.154.545 \\
		\hline
		Kerosene  & \$1.245.455 \\
		\hline
		Total     & \$4.400.000\\
		\hline
	\end{tabular}
\end{center}

En ese caso se puede ver que la ganancia es menor, esto se debe a que los costos fijos de Vehículos y Kerosene siguen estando, quedan horas extra en las cuales se podría producir mas combustible, en el escenario anterior era combustible de avión que llegaba a cubrir el costo fijo de su embalaje y generaba unos $\$950.000$ extra, el valor que diferencia los resultados.


\section{}
\textbf{Y si hubiese aumentado lo máximo posible la producción de los otros productos? Suponer que se cierra el
	sector de embalaje de combustibles para aviones.}

Para eso formulamos el siguiente LP:

\begin{align*}
	\text{Max} \quad & 2900 X_v + 1200 X_k - 11\,500\,000 \\
	\text{Subject to} \quad
	& 5 X_v + 3 X_k \leq 38\,000 \quad \text{(restricciones sobre el refinado)} \\
	& 10 X_v + 6 X_k \leq 80\,000 \quad \text{(restricciones sobre el fraccionado)} \\
	& 2 X_v \leq 6\,000  \quad \text{(restricciones sobre el embalaje de combustible para vehículos)}\\
	& X_k \leq 7\,000  \quad \text{(restricciones sobre el embalaje de kerosene)}\\
	& X_v \geq 0,\quad X_k \geq 0
\end{align*}

donde $X_v$ son miles de litros de combustible para vehículos y  $X_k$ son miles de litros de kerosene. Los coeficientes de la función objetivo es el precio de venta cada 1000 litros de combustible menos los costos variables $C_{ij}$ mencionados anteriormente. Los costos fijos se le restan directamente a la función objetivo.

El valor óptimo para la producción es de $X_v = 3000, X_k = 7000$. Expresado en miles de litros de combustible a producir. La ganancia de la empresa aumentaría con estos nuevos valores:
\clearpage
\begin{center}
	\textbf{Ganancias totales}
	\vspace{3mm}
	
	\begin{tabular}{| c | c |}
		\hline
		&           Ganancia    \\
		\hline
		Aviones   & \$0  \\
		\hline
		Vehículos & \$3.533.333 \\
		\hline
		Kerosene  & \$2.066.667 \\
		\hline
		Total     & \$5.600.000\\
		\hline
	\end{tabular}
\end{center}

\section{} %4
\section{} %5

\section*{Diccionario óptimo}

Dado el método simplex en su forma matricial:
\begin{tabular}{c}
	$X_B = B^{-1}b - B^{-1}A_NX_N$ \\
	\hline
	$z = c_B B^{-1} b + (c_N - c_B B^{-1} A_N) X_N$
\end{tabular}

Nuestro diccionario óptimo va a tener a $X_a$, $X_v$, y $X_k$ como variables básicas porque toman valores no nulos en la solución óptima.

También vemos que las restricciones $10 X_v + 6 X_k + 20 X_a \leq 80000$ (2) y $X_k \leq 7000$ (4) se cumplen por desigualdad estricta, entonces sus variables de holgura correspondientes van a ser mayores a 0 en la solución óptima.
Siendo $X_4, \dots, X_8$ las variables de holgura correspondientes a las restricciones $1, \dots, 5$ respectivamente, esto significa que $X_5$ y $X_7$ son variables básicas.

Entonces tenemos en la base óptima: 

\[B = \begin{pmatrix}
	5  & 3 & 10 & 0 & 0 \\
	10 & 6 & 20 & 1 & 0 \\
	2  & 0 & 0  & 0 & 0 \\
	0  & 1 & 0  & 0 & 1 \\
	0  & 0 & 4  & 0 & 0
\end{pmatrix}\]
Entonces su inversa es:
\[B^{-1} = \begin{pmatrix}
	0  & 0 &\frac{1}{2} & 0 & 0 \\
	\frac{1}{3} & 0 & \frac{-5}{6} & 0 & \frac{-5}{6} \\
	0  & 0 & 0  & 0 & \frac{1}{4} \\
	-2 & 1 & 0  & 0 & 0 \\
	\frac{-1}{3}  & 0 & \frac{5}{6}  & 1 & \frac{5}{6}
\end{pmatrix}\]

También tenemos $c_B = \begin{pmatrix} 5900 & 2900 & 1200 & 0 & 0 \end{pmatrix}, c_N = \begin{pmatrix} 0 & 0 & 0 \end{pmatrix}$, $A_N = \begin{pmatrix}
	1 & 0 & 0 \\
	0 & 0 & 0 \\
	0 & 1 & 0 \\
	0 & 0 & 0 \\
	0 & 0 & 1 \\
\end{pmatrix}$, $b = \begin{pmatrix} 
3800 \\
80000 \\
6000 \\
7000 \\
4000 
\end{pmatrix}$

\section{} %6
\section{} %7
\section{} %8
\section{} %9
\textbf{Por otro lado, el gerente de compras propone cambiar algunos proveedores, lo que permitiría bajar el costo de la materia prima del aceite para vehículos de \$1000 a \$800 por cada 1000 litros procesados. ¿Cambiaría el plan de producción óptimo? Si es así, dar la nueva planificación óptima.}


\section{} %10
\section{} %11
\section{} %12
\end{document}
