\documentclass[10pt,a4paper]{article}
\usepackage[utf8]{inputenc}
\usepackage[spanish]{babel}
\usepackage{a4wide} % márgenes un poco más anchos que lo usual
\usepackage{caratula}

\begin{document}

\titulo{TP1}

\fecha{\today}

\materia{Introdicción a la Investigación Operativa y Optimización}

\integrante{Laks, Joaquín}{425/22}{laksjoaquin@gmail.com}

\maketitle

\section*{Datos}

\begin{center}
\textbf{Horas por 1000 litros de combustible}

\begin{tabular}{| c || c | c | c |}
	\hline
						& Refinado & Fraccionado & Embalaje \\
	\hline
	Aviones   & 10       & 20          & 4 \\
	\hline
	Vehículos & 5        & 10          & 2 \\
	\hline
	Keronsene & 3        & 6           & 1 \\
	\hline
\end{tabular}

\vspace{5mm}
\textbf{Tiempos y gastos fijos}

\begin{tabular}{| c || c | c |}
	\hline
										& Capacidad Mensual & Gasto Fijo \\
	\hline
	Refinado           & 38.000 horas      & \$5.000.000 \\
	\hline
	Fraccinoado        & 80.000 horas      & \$5.000.000 \\
	\hline
	Embalaje Aviones   & 4.000 horas       & \$2.000.000 \\
	\hline
	Embalaje Vehículos & 6.000 horas       & \$1.000.000 \\
	\hline
	Embalaje Keronsene & 7.000 horas       & \$500.000 \\
	\hline
\end{tabular}

\vspace{5mm}

	\textbf{Costos por 1000 litros}

	\begin{tabular}{| c || c | c | c | c | c |}
		\hline
							& Precio de venta & Materia prima & Refinado & Fraccionado & Embalaje\\
		\hline
		Aviones   & \$16.000        & \$4.000       & \$4.100  & \$1.000   & \$1.000   \\
		\hline
		Vehículos & \$8.000         & \$1.000       & \$3.000  & \$600     & \$500   \\
		\hline
		Keronsene & \$4.000         & \$500         & \$1.500  & \$400     & \$400 \\
		\hline
	\end{tabular}
\end{center}

\section{}
\textbf{Calcular la ganancia o pérdida (prorrateando los gastos fijos) de cada producto que se obtuvo en el mes
anterior (cuando se produjeron 500.000 litros de combustible para aviones, 3.000.000 de combustible para
vehículos y 6.000.000 litros de kerosene) y la ganancia (o pérdida) total de la compañía.}

\vspace{5mm}

Primero calculamos los costos y ganancias de cada tipo de combustible sin tener en cuenta los costos fijos. Para saber cuánto costó en la etapa $j$ producir $l_i$ litros del combustible $i$ si sabemos que cada 1000 litros nos cuesta $c_{ij}$, hacemos una regla de tres simple y nos queda un gasto de $\frac{l_i}{1000} c_{ij}$. También calculamos su precio de venta y nos quedamos con un total de ganancia neta sin tener en cuenta todavía los costos fijos.

\begin{center}
	\textbf{Ingresos y costos no fijos del último mes}

	\begin{tabular}{| c || c | c | c | c | c || c |}
		\hline
							& Precio de venta & Materia prima & Refinado     & Fraccionado & Embalaje  & Ganancia\\
		\hline
		Aviones   & \$8.000.000     & \$2.000.000   & \$2.050.000  & \$500.000   & \$500.000 & \$2.950.000  \\
		\hline
		Vehículos & \$24.000.000    & \$3.000.000   & \$9.000.000  & \$1.800.000 & \$1.500.000 & \$8.700.000\\
		\hline
		Keronsene & \$24.000.000    & \$3.000.000   & \$9.000.000  & \$2.400.000 & \$2.400.000 & \$7.200.000 \\
		\hline
	\end{tabular}
\end{center}

Para prorratear los gastos fijos, calculamos cuántas horas se usaron para cada etapa del proceso en cada tipo de combustible de la misma manera que calculamos los gastos.
\begin{center}
	\textbf{Horas gastadas}

	\begin{tabular}{| c || c | c | c |}
		\hline
		&           Refinado & Fraccionado & Embalaje \\
		\hline
		Aviones   & 5.000    & 10.000      & 2.000    \\
		\hline
		Vehículos & 15.000   & 30.000      & 6.000    \\
		\hline
		Kerosene  & 9.000    & 36.000      & 6.000    \\
		\hline
		\hline
		Total     & 29.000   & 76.000      & 14.000   \\
		\hline
	\end{tabular}
\end{center}

Ahora prorrateamos los costos fijos compartidos, le sumamos el costo fijo de embalaje y tenemos los últimos costos.

\begin{center}
	\textbf{Costos prorrateados}

	\begin{tabular}{| c || c | c | c || c |}
		\hline
		&           Refinado    & Fraccionado & Embalaje    & Total \\
		\hline
		Aviones   & \$862.069   & \$657.895   & \$2.000.000 & \$3.519.964  \\
		\hline
		Vehículos & \$2.586.206 & \$1.973.684 & \$1.000.000 & \$5.559.891  \\
		\hline
		Kerosene  & \$1.551.724 & \$2.368.421 & \$500.000   & \$4.420.145 \\
		\hline
	\end{tabular}
\end{center}

Le restamos estos costos a las ganancias que teníamos antes y tenemos las ganancias totales.

\begin{center}
	\textbf{Ganancias totales}

	\begin{tabular}{| c | c |}
		\hline
		&           Ganancia    \\
		\hline
		Aviones   & -\$569.964  \\
		\hline
		Vehículos & \$3.140.109 \\
		\hline
		Kerosene  & \$2.779.855 \\
		\hline
		Total     & \$5.350.000\\
		\hline
	\end{tabular}
\end{center}

Vemos que la empresa en total da ganancia, pero el combustible para aviones pérdida.

\section{}
\textbf{Si la empresa no hubiese producido combustible para aviones manteniendo en los mismos valores los otros productos, ¿la ganancia de la compañía habría sido mejor? Suponer que se cierra el sector de embalaje de combustibles para aviones.}

En ese caso, las ganancias netas sin los costos fijos de Vehículos y Kerosene serían las mismas, y al total de la empresa le restamos todos los costos fijos sin contar el embalaje de combustible para aviones. Haciendo la cuenta, nos quedaría una ganancia de \$4.400.000, menor a la ganancia produciendo aviones.

\end{document}
