\documentclass[10pt,a4paper]{article}
\usepackage[utf8]{inputenc}
\usepackage[spanish]{babel}
\usepackage{a4wide} % márgenes un poco más anchos que lo usual
\usepackage{caratula}
\usepackage{amsmath}
\usepackage{amssymb}

\begin{document}

	\titulo{TP2}

	\fecha{\today}

	\materia{Introducción a la Investigación Operativa y Optimización}

	\integrante{Laks, Joaquín}{425/22}{laksjoaquin@gmail.com}
	\integrante{Szabo, Jorge}{1683/21}{jorgecszabo@gmail.com}
	\integrante{Wilders Azara, Santiago}{350/19}{santiago199913@gmail.com}

	\maketitle

	Basado en Miller, Tucker, y Zemlin

	Dada un cliente $i$ definimos $D_i$ como los clientes a distancia menor a $dist\_max$ de $i$.

	\[
		x_{ij} = \begin{cases}
			1 & \text{si desde el cliente $v_i$ el camión se mueve al cliente $v_j$}\\
			0 & \text{c.c.}
		\end{cases}
	\]
	\[
		u_{i} = \text{posición del cliente $i$ en el circuito del camión (no importa cuando el camión no pasa)}
	\]
	\[
		b_{ij} = \begin{cases}
			1 & \text{si se envió un repartidor en bicicleta desde $v_i$ hasta $v_j$}\\
			0 & \text{c.c}\\
		\end{cases}
	\]

	$b_{ij}$ solo está definida para $j \in D_i$, en la formulación queda más cómodo escribir como si estuviera definido para todas las parejas pero se pueden interpretar $r_{ij}$ inválidos como constantes 0.

	También contamos con el dato de entrada:

	\[
		r_{i} = \begin{cases}
			1 & \text{si al cliente $v_i$ se le entrega un producto que necesita refrigeración}\\
			0 & \text{c.c}\\
		\end{cases}
	\]

	Con $n = cant\_clientes$, buscamos:

	\[
		\text{Min } \sum_{ \begin{tabular}{c}
				$v_i, v_j \in V$ \\
				$ i \neq j$
		\end{tabular}
		} c_{ij} x_{ij} + costo\_repartidor\,b_{ij}
	\]

	s.a.

	\[
	\begin{array}{l l l}
		\sum_{j \neq i} x_{ji} + \sum_{j \neq i} b_{ji} = 1 & \forall v_i \in V & \text{a toda ciudad se entra una vez, por camión o bicicleta} \\
		\\
		\sum_{j \neq i} x_{ij} = \sum_{j\neq i} x_{ji} & \forall v_i \in V & \text{si se entró en camión, se sale por camión} \\
		\\
		M(1 - \sum_{j \neq i} b_{ji}) \geq \sum_{j \neq i} b_{ij} + \sum_{j \neq i} x_{ij} & \forall v_i \in V, M \geq |V| & \text{si se entra en bicicleta, no se sale de ninguna forma} \\
		\\
		\sum_{j \in D_i} b_{ij} r_j \leq 1 & \forall v_i \in V & \text{ningún repartidor tiene más de un refrigerado} \\
		\\
		u_i - u_j + (n - 1) x_{ij} \leq n - 2 & \forall v_i \neq v_j \in V - \{a\} & \text{continuidad} \\
		\\
		u_1 = 0, 1 \leq u_i \leq n-1 & \forall v_i \in V - \{v_1\} \\
		\\
		x_{ij}, b_{ij} \in \{0,1\}, u_i \in \mathbb{Z}_{\geq 0}
	\end{array}
	\]
\end{document}
