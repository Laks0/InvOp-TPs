\documentclass[10pt]{article}
\usepackage[a4paper, margin=1in]{geometry}
\usepackage[utf8]{inputenc}
\usepackage[spanish]{babel}
\usepackage{caratula}
\usepackage{amsmath}
\usepackage{amssymb}
\usepackage{hyperref}
\hypersetup{
    colorlinks=true,
    linkcolor=blue,
    filecolor=magenta,      
    urlcolor=cyan,
    pdftitle={Overleaf Example},
    pdfpagemode=FullScreen,
	}

\begin{document}

	\titulo{TP2}

	\fecha{\today}

	\materia{Introducción a la Investigación Operativa y Optimización}

	\integrante{Laks, Joaquín}{425/22}{laksjoaquin@gmail.com}
	\integrante{Szabo, Jorge}{1683/21}{jorgecszabo@gmail.com}
	\integrante{Wilders Azara, Santiago}{350/19}{santiago199913@gmail.com}

	\maketitle

\section{Modelos}



\subsection{Modelo para la metodología actual}

La metodología actual es idéntico al problema del TSP clásico. Usamos el modelo de Miller, Tucker, y Zemlin para simularlo y como base para las otras metodologías.

\subsection{Primer modelo con repartidores}

Basado en el modelo anterior, agregamos variables para distinguir cuándo un cliente fue visitado por repartidor a bicicleta y cuándo fue visitado por el camión.

	\vspace{5mm}

	Dado un cliente $i$ definimos $D_i$ como los clientes a distancia menor a $dist\_max$ de $i$.

	\[
		x_{ij} = \begin{cases}
			1 & \text{si desde el cliente $v_i$ el camión se mueve al cliente $v_j$}\\
			0 & \text{c.c.}
		\end{cases}
	\]
	\[
		u_{i} = \text{posición del cliente $i$ en el circuito del camión (no importa cuando el camión no pasa)}
	\]
	\[
		b_{ij} = \begin{cases}
			1 & \text{si se envió un repartidor en bicicleta desde $v_i$ hasta $v_j$}\\
			0 & \text{c.c}\\
		\end{cases}
	\]

	$b_{ij}$ solo está definida para $j \in D_i$, por claridad en la formulación está escrito como si estuviera definido para todas las parejas pero se pueden interpretar $b_{ij}$ inválidos como constantes 0.

	También contamos con el dato de entrada:

	\[
		r_{i} = \begin{cases}
			1 & \text{si al cliente $v_i$ se le entrega un producto que necesita refrigeración}\\
			0 & \text{c.c}\\
		\end{cases}
	\]

	Con $n = cant\_clientes$, buscamos:

	\[
		\text{Min } \sum_{ \begin{tabular}{c}
				$v_i, v_j \in V$ \\
				$ i \neq j$
		\end{tabular}
		} c_{ij} x_{ij} + costo\_repartidor\,b_{ij}
	\]

	s.a.

	\[
	\begin{array}{l l l}
		\sum_{j \neq i} x_{ji} + \sum_{j \neq i} b_{ji} = 1 & \forall v_i \in V & \text{a toda ciudad se entra una vez, por camión o bicicleta} \\
		\\
		\sum_{j \neq i} x_{ij} = \sum_{j\neq i} x_{ji} & \forall v_i \in V & \text{si se entró en camión, se sale por camión} \\
		\\
		M(1 - \sum_{j \neq i} b_{ji}) \geq \sum_{j \neq i} b_{ij} + \sum_{j \neq i} x_{ij} & \forall v_i \in V, M \geq |V| & \text{si se entra en bicicleta, no se sale de ninguna forma} \\
		\\
		\sum_{j \in D_i} b_{ij} r_j \leq 1 & \forall v_i \in V & \text{ningún repartidor tiene más de un refrigerado} \\
		\\
		u_i - u_j + (n - 1) x_{ij} \leq n - 2 & \forall v_i \neq v_j \in V - \{v_1\} & \text{continuidad} \\
		\\
		u_1 = 0, 1 \leq u_i \leq n-1 & \forall v_i \in V - \{v_1\} \\
		\\
		x_{ij}, b_{ij} \in \{0,1\}, u_i \in \mathbb{Z}_{\geq 0}
	\end{array}
	\]
	
	\subsection{Modelo con restricciones agregadas}
	Este modelo extiende el anterior con las siguientes restricciones adicionales:
	\begin{itemize}
		\item Si se contrata un repartidor en una parada de camión determinada, este debe realizar al menos cuatro entregas.
		\item Hay un conjunto de clientes $E$ que deben ser visitados exclusivamente por un camión. Es decir que sus paquetes no pueden ser repartidos por un repartidor en bicicleta.
	\end{itemize}
	
	Se extiende el modelo presentado en la sección anterior con las siguientes restricciones.
	
		\[
	\begin{array}{l l l}
		4(1 - b_{ij}) + \sum_{v_k \in D_i} b_{ik} \geq 4 & \forall v_i, v_j \in V \times V & \text{Si se contrata un repartidor, este pasa por al menos 4 clientes} \\
		\\
		\sum_{v_i \in V}  b_{ij} = 0 & \forall v_j \in E  & \text{Los clientes exclusivos solo son visitador por camión} \\
	\end{array}
	\]
	
	No hay cambios en las variables, función objetivo o las demás restricciones.

	\section{Experimentación}

	Usamos un generador de instancias aleatorias para generar mapas donde algunos caminos son posibles y otros no, asignando distanias y costos al azar, así como también qué clientes necesitan refrigeración o tienen que vistiarse con camión en el último modelo. También sacamos datos de posiciones de ciudades de \url{https://people.sc.fsu.edu/~jburkardt/datasets/cities/cities.html}, usamos las distancias y las modificamos aleatoriamente para crear una instancia con costos proporcionales. De esta página tomamos las instancias \texttt{wg59} con 59 ciudades de Alemania Oriental, y \texttt{usca321} con 321 ciudades de Estados Unidos y Canadá. Nos interesó crear instancias con información geográfica para tener una distribución orgánica de los puntos, \texttt{usca321} en particular, además de ser una instancia grande, es interesante porque la densidad de los puntos no es uniforme.

	Para decidir si un viaje entre dos puntos es posible agregamos aleatoriamente aristas entre los puntos, llamamos \textit{ralas} a las instancias donde la probabilidad de agregar una arista es de $0.3$ y \textit{densas} cuando la probabilidad es de $0.9$. Siempre asegurándonos de que el grafo resultante quede conexo.

	También con los puntos creamos instancias donde la proporción de refrigerados es del $70\%$ y otras donde es el $10\%$. Hicimos lo mismo con los exclusivos.

	Por último también agregamos instancias donde la distancia máxima del repartidor era muy corta o muy larga, y donde el repartidor era muy barato o muy caro. Por defecto la distancia máxima del repartidor la tomamos como la mediana de todas las distancias, pero para representar una distancia máxima más corta usamos el primer quintil, y para una más larga el tercero. Hicimos lo mismo con los costos para representar cuándo el repartidor es más caro o más barato.

\end{document}
